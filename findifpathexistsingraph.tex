% Created 2025-11-07 Fri 21:29
% Intended LaTeX compiler: pdflatex
\documentclass[11pt]{article}
\usepackage[utf8]{inputenc}
\usepackage[T1]{fontenc}
\usepackage{graphicx}
\usepackage{longtable}
\usepackage{wrapfig}
\usepackage{rotating}
\usepackage[normalem]{ulem}
\usepackage{amsmath}
\usepackage{amssymb}
\usepackage{capt-of}
\usepackage{hyperref}
\author{Joel Boynton}
\date{\today}
\title{FindIfPathExistsInGraph}
\hypersetup{
 pdfauthor={Joel Boynton},
 pdftitle={FindIfPathExistsInGraph},
 pdfkeywords={},
 pdfsubject={},
 pdfcreator={Emacs 28.3 (Org mode 9.7.29)}, 
 pdflang={English}}
\begin{document}

\maketitle
\clearpage
\section{Problem Description}
\label{sec:orgc846410}
For this problem we are given a bidirectional graph with an 'n' number of vertices. Each pair of vertices should have no more than one edge connecting the two. There should be no vertices with an edge to itself. In this problem we will be presented with a source and destination vetex which we will need to determine whether there is a valid path between the two. If the path between the two is valid (following the aforementioned rules) then our program should return true, and false otherwise.
\section{Solution Description}
\label{sec:org1bb6d4a}
For the solution I decided on a Breadth-First Search (BFS) approach. In this approach we will start with our \emph{source} vertex. Which will serve as our entrance in to the graph. We move through the graph by first checking the immediate neighbors to the source. After this we then go through each of \emph{their} neighbors navigating away from the source layer by layer. If, during this exploration, we come across the destination we are looking for, then we can determine that a path exists. If we go through each connected vertex, never encountering the desired destination, then we can deduce that no path exists.
\section{Initial Attempt to Code the Solution}
\label{sec:org9c9414b}
For this problem we will use a few different data structures: \emph{Adjaceny lists, a queue, and a visited set}. I started by getting the trivial case out of the way (source == destination) and then moved on to building my logic. I start with building my adjaceny list (vector<vector<int>> adj) which will be of size \emph{n}. This will then iterate through the edges list for each edge \emph{{[}u, v]}. For each adj[u], I push back into v (adding v to u's list) and vice versa. After this I initialized my \emph{BFS} by creatin a queue \emph{q} and a boolean vector. I add the source to the queue, and the mark it as visited. While the queue is not empty I loop through each vertex, iterating through each neighbor until I find the destination. If not then \emph{destination} returns false.
\section{Testing Description}
\label{sec:org65a8338}
Testing done in LeetCode
\section{Code \& Testing}
\label{sec:org59eec6e}
\begin{verbatim}
#include <vector>
#include <queue>
using namespace std;


class Solution{
public:
  bool validPath(int n, vector<vector<int>>& edges, int source, int destination) {
    if (source == destination) {
      return true;
    }

    vector<vector<int>> adj(n);
    for(const auto& edge : edges) {
      int u = edge[0];
      int v = edge[1];
      adj[u].push_back(v);
      adj[v].push_back(u);
    }

    queue<int> q;
    vector<bool> visited(n, false);

    q.push(source);
    visited[source] = true;

    while(!q.empty()) {
      int u = q.front();
      q.pop();

      for(int v : adj[u]) {
        if (v == destination){
          return true;
        }

        if(!visited[v]){
          visited[v] = true;
          q.push(v);
        }
      }
    }

    return false;
  }
};
\end{verbatim}

\begin{figure}[H]
\centering
\includegraphics[scale=0.3]{~/Documents/College/Fall25/ECGR3180/HomeWork/LeetCode/FindIfPathExists/Results.png}
\end{figure}

\begin{figure}[H]
\centering
\includegraphics[scale=0.2]{~/Documents/College/Fall25/ECGR3180/HomeWork/LeetCode/FindIfPathExists/Complexity.png}
\end{figure}
\end{document}
