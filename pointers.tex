% Created 2025-07-07 Mon 09:37
% Intended LaTeX compiler: pdflatex
\documentclass[11pt]{article}
\usepackage[utf8]{inputenc}
\usepackage[T1]{fontenc}
\usepackage{graphicx}
\usepackage{longtable}
\usepackage{wrapfig}
\usepackage{rotating}
\usepackage[normalem]{ulem}
\usepackage{amsmath}
\usepackage{amssymb}
\usepackage{capt-of}
\usepackage{hyperref}
\author{Joel Boynton}
\date{\today}
\title{Pointers}
\hypersetup{
 pdfauthor={Joel Boynton},
 pdftitle={Pointers},
 pdfkeywords={},
 pdfsubject={},
 pdfcreator={Emacs 28.3 (Org mode 9.7.29)}, 
 pdflang={English}}
\begin{document}

\maketitle
\href{ecgr2104.org}{ECGR2104: Notes}
\section{Defintion}
\label{sec:org3802979}
\textbf{Pointer:} a variable that stores the memory of another variable
\begin{verbatim}
int x = 10;
int* p = &x;
\end{verbatim}
\begin{quote}
the above allows for us to store the memory location of x on p
\end{quote}
\section{Properties}
\label{sec:org6276eb3}
\textbf{Declaration}

\begin{itemize}
\item \textbf{int*\_\_\_}     // Is used to emphasize that the variable being declared is a pointer of some type (i.e. int, double, short)

\item \textbf{= \&\_\_\_}   // Returns/pulls the memory address of a specific variable
\end{itemize}

\begin{quote}
In combination \textbf{int*p = \&x;} returns the address of x and stores it within p
\end{quote}
\textbf{The Dereference Operator} (*)
\begin{verbatim}
#include <iostream>
using namspace std;

int x = 10;
int* p = &x; //*p = x = 10
cout << *p; //outputs 10
\end{verbatim}
\begin{quote}
The ``dereferenced'' \textbf{p} can now reassign \textbf{x} in the opposite way as well; as they share the same address.
\end{quote}
\begin{verbatim}
int x = 10;
int* p = &x; // *p = x = 10
*p = 15;    // *p = 15 = x
\end{verbatim}
\end{document}
